% !TEX spellcheck = off
\documentclass[12pt, a4paper]{article}
\usepackage[english]{babel}
\usepackage{amsmath, amssymb, amsfonts}
\usepackage{graphicx}
\usepackage{hyperref}
\usepackage{geometry}
\usepackage{enumitem}
\usepackage[normalem]{ulem}
\usepackage{setspace}
\onehalfspacing
\geometry{left=2.5cm, right=2.5cm, top=2.5cm, bottom=2.5cm}
\hypersetup{
    unicode=true,
    pdftitle={Alternative Gravity Hypothesis as an Effect of Differential Universe Expansion},
    pdfauthor={Leonid Serebrennikov},
    pdfsubject={Theoretical Physics, Cosmology},
    pdfkeywords={alternative gravity, cosmological expansion, dark matter, Hubble parameter, falsifiability}
}

\title{Alternative Gravity Hypothesis as an Effect of Differential Universe Expansion}
\author{Leonid Serebrennikov\thanks{Corresponding author: rubikkon@gmail.com}}
\date{\today}

\begin{document}

\maketitle

\begin{abstract}
    In the standard model, gravity is described either as a fundamental interaction (Newton's law of universal gravitation) or as a geometric property of spacetime (Einstein's general theory of relativity). The hypothesis proposed here considers observable gravitational attraction as a consequence of uneven cosmological expansion. Massive objects, possessing higher energy density, cause more intense local expansion of space, leading to effective repulsion of less massive bodies toward zones of increased expansion. The paper outlines the main postulates of the model, its potential implications for the dark matter problem, and proposes specific paths for experimental falsification.

    \textbf{Keywords:} alternative gravity, cosmological expansion, dark matter, differential expansion, falsifiability.
\end{abstract}

\section*{Note on Language}
A Russian version of this article is available at:\\
\url{https://github.com/rubikkon/new-gravity-definition}\\
The Russian version contains the original formulation of this hypothesis.

\section{Introduction}
\label{sec:intro}

Cosmological expansion is empirically confirmed: the Hubble–Lemaître law \cite{hubble}, Type Ia supernova data, and anisotropy of the cosmic microwave background (missions COBE, WMAP, Planck \cite{planck}). The acceleration of expansion in modern cosmology is attributed to dark energy (the $\Lambda$-term in the Friedmann equations). This work proposes a hypothesis according to which observable gravitational attraction on the scales of the Solar System and galaxies can be reinterpreted as a macroscopic manifestation of local variations in the rate of this expansion.

\section{Main Postulate}
\label{sec:postulate}

The hypothesis postulates a dependence of the local expansion rate (derivative of the scale factor) on the local energy density $\rho$ in a given region of spacetime:
\begin{equation}
    \label{eq:main}
    \dot{H}_{\text{loc}} = H_0 + \kappa \rho,
\end{equation}
where $\dot{H}_{\text{loc}}$ is the local derivative of the Hubble parameter (or scale factor), $H_0$ is the global Hubble constant characterizing the background expansion of the Universe, $\kappa$ is a phenomenological coefficient linking energy density to the contribution to expansion.

The resulting gradient of $\dot{H}_{\text{loc}}$ creates an effective force directed from regions of lower energy density to regions of higher density. At the macroscopic level, this phenomenon is phenomenologically equivalent to Newtonian attraction, although its microscopic mechanism has an opposite nature (effective "repulsion" from zones of lesser expansion to zones of greater expansion).

\section{Potential Implications: The Dark Matter Problem}
\label{sec:consequences}

This model offers an alternative explanation for anomalies in galactic rotation curves, usually attributed to the presence of dark matter. According to the hypothesis, increased expansion in the central regions of galaxies (which have higher density of stars and gas) may affect the dynamics of peripheral stars, increasing their tangential velocity. In the standard interpretation, this is observed as the presence of invisible mass creating additional gravitational field.

\section{Possible Paths for Falsification}
\label{sec:falsification}

The following experimental and observational methods can be used to test the proposed hypothesis:

\begin{enumerate}
    \item \textbf{Precision astrometry.} Projects such as the \textit{Gaia} mission \cite{gaia} or future space interferometers could reveal small deviations in the motion of Solar System bodies or binary stars from the predictions of General Relativity. According to the hypothesis, these deviations should correlate with local values of the $H_0$ parameter.

    \item \textbf{Gravitational wave analysis.} Detectors LIGO \cite{ligo}, Virgo, KAGRA, and the future LISA mission might detect weak cosmological corrections to the phase and amplitude of signals from mergers of compact objects at large redshifts, caused by the dependence of gravitational interaction on local energy density.

    \item \textbf{Cosmological surveys.} Data from missions such as \textit{Euclid}, \textit{Roman Space Telescope}, or LSST will allow investigation of possible correlations between the distribution of visible matter and local variations of the expansion parameter on scales of 10–100 Mpc.

    \item \textbf{Laboratory experiments.} High-precision measurements using atomic clocks or modified Cavendish-type setups could be aimed at detecting possible dependence of the effective gravitational "constant" $G$ on ambient density or its potential temporal variations.
\end{enumerate}

\section{Conclusion and Discussion}
\label{sec:conclusion}

The presented hypothesis considers gravity not as a fundamental interaction, but as a consequence of differential cosmological expansion modulated by the distribution of energy density. This offers a path toward conceptual unification of gravity and cosmological acceleration phenomena (as opposed to the General Relativity approach \cite{einstein}) and potentially eliminates the need for hypothetical dark matter in explaining galactic kinematics.

However, the hypothesis requires further extensive development, including:
\begin{itemize}
    \item Mathematical formalization within a modified metric or effective field theory framework.
    \item Quantitative derivation of Newtonian and post-Newtonian limits.
    \item Strict consistency with the complete set of high-precision tests of General Relativity in the Solar System.
\end{itemize}

At present, the hypothesis remains speculative and lacks direct empirical confirmation. Its value lies in proposing a new research direction and specific, potentially falsifiable predictions for the next generation of astrophysical and laboratory experiments.

\section*{Acknowledgments}
The author thanks colleagues from the Department of Theoretical Physics for valuable discussions. This work was performed as a personal research initiative.

\vspace{1cm}

\noindent \textbf{Version Information:} A Russian version of this work is available in the repository: \url{https://github.com/rubikkon/new-gravity-definition}

\begin{thebibliography}{99}
    \bibitem{hubble} Hubble, E.~P. A relation between distance and radial velocity among extra-galactic nebulae. \textit{Proceedings of the National Academy of Sciences}, 15(3):168--173, 1929.

    \bibitem{planck} Planck Collaboration. Planck 2018 results. VI. Cosmological parameters. \textit{Astronomy \& Astrophysics}, 641:A6, 2020.

    \bibitem{gaia} Gaia Collaboration. The Gaia mission. \textit{Astronomy \& Astrophysics}, 595:A1, 2016.

    \bibitem{ligo} Abbott, B.~P. et al. Observation of gravitational waves from a binary black hole merger. \textit{Physical Review Letters}, 116(6):061102, 2016.

    \bibitem{einstein} Einstein, A. The foundation of the general theory of relativity. \textit{Annalen der Physik}, 354(7):769--822, 1916.
\end{thebibliography}

\end{document}