\documentclass[12pt,a4paper]{article}
\usepackage[utf8]{inputenc}
\usepackage[T2A]{fontenc}
\usepackage[russian]{babel}
\usepackage{amsmath}
\usepackage{amssymb}
\usepackage{hyperref}
\usepackage{indentfirst} % красная строка

\hypersetup{
    colorlinks=true,
    linkcolor=blue,
    citecolor=blue,
    urlcolor=blue
}

% === ЗАГОЛОВОК (двуязычный для большей видимости на ai.vixra.org) ===
\title{К вопросу о природе объектов, наблюдаемых как «чёрные дыры» \\
       {\small On the Question of the Nature of Objects Observed as ``Black Holes''}}

% === ЗАПОЛНИТЕ СВОИ ДАННЫЕ ЗДЕСЬ (вы потом добавите) ===
\author{Ваше Имя \\
        Аффилиация (или «независимый исследователь») \\
        \texttt{ваш@email.ru} \\[0.5em]
        \small{(AI-assisted preprint для ai.vixra.org)}}
\date{\today}

\begin{document}

\maketitle

\begin{abstract}
В работе предлагается модель, объясняющая наблюдаемые характеристики компактных объектов в центрах галактик (в частности, M87* и Sgr A*) без привлечения концепции горизонта событий и гравитационного захвата света. Основой модели является постулат о том, что гравитация воздействует исключительно на частоту излучения (красное смещение) и не способна изменить траекторию движения фотонов или остановить их. Наблюдаемая структура «светящееся кольцо + тёмная центральная область» объясняется комбинацией отражения излучения от реальной поверхности объекта и эффектов космологического расширения (Хаббла), действующих на свет в процессе его распространения до наблюдателя.
\end{abstract}

\begin{abstract}[English]
This paper proposes a model explaining the observed characteristics of compact objects in galactic centres (in particular M87* and Sgr~A*) without invoking the concept of an event horizon or gravitational light capture. The model is based on the postulate that gravity affects exclusively the frequency of radiation (gravitational redshift) and cannot change the trajectory of photons or stop them. The observed structure ``luminous ring + dark central region'' is explained by a combination of radiation reflection from the object's real surface and the effects of cosmological expansion (Hubble law) acting on light during its propagation to the observer.
\end{abstract}

\textbf{Ключевые слова:} чёрные дыры, горизонт событий, гравитационное красное смещение, эффект Хаббла, отражение света, альтернативная модель гравитации

\textbf{Keywords:} black holes, event horizon, gravitational redshift, Hubble expansion, light reflection, alternative gravity model

\section{Исходные положения}

\subsection{Гравитация и свет}
Гравитационное поле напрямую не взаимодействует со светом как с движущимся объектом в смысле изменения скорости или направления движения. Единственным эффектом гравитации на излучение является изменение его частоты (гравитационное красное смещение). Фотон, покидающий область с сильным гравитационным полем, теряет энергию, что регистрируется как увеличение длины волны. Перераспределение энергии на расстоянии.

\subsection{Наличие поверхности}
Исследуемые объекты (традиционно именуемые «чёрными дырами») представляют собой компактные тела, обладающие материальной поверхностью. Термин «горизонт событий» рассматривается как математическая абстракция, не имеющая физического воплощения.

\section{Механизм формирования наблюдаемого изображения}

\subsection{Источник излучения}
Излучение, регистрируемое от объекта, формируется в результате падения внешнего света (от окружающих звёзд, межзвёздной среды или самого объекта) на его поверхность и последующего отражения.

\subsection{Красное смещение при отражении}
В момент отражения от поверхности, находящейся в сильном гравитационном поле, излучение испытывает красное смещение. Величина смещения зависит от гравитационного потенциала в точке отражения.

\subsection{Влияние расширения Хаббла на этапе распространения}
После отражения свет движется к наблюдателю. За время его распространения космологическое расширение (закон Хаббла) воздействует как на само излучение, так и на геометрические характеристики объекта-источника. Расширение приводит к «растяжению» пространственной структуры, с которой было испущено излучение.

\subsection{Трансформация пучка в кольцевую структуру}
Вследствие растяжения области отражения за время полёта света, первоначально компактный пучок отражённого от центральной части поверхности излучения преобразуется в кольцевую структуру к моменту достижения наблюдателя. Излучение, отражённое от краевых зон, формирует внешнюю часть кольца. Излучение из центральной области оказывается «размазанным» по кольцу, что создаёт иллюзию отсутствия сигнала из центра.

\subsection{Природа тёмной центральной области}
Чёрная область внутри кольца не является зоной отсутствия излучения или «тенью» от невидимого объекта. Это область, излучение из которой пришло к наблюдателю в трансформированном виде (в составе кольца) и будет регистрироваться как часть кольца при последующих наблюдениях с иными временными задержками.

\section{Наблюдательные следствия и предсказания}

\subsection{Зависимость изображения от расстояния}
Эффект трансформации точечного отражения в кольцо является функцией расстояния до объекта. Чем дальше объект, тем сильнее сказывается расширение Хаббла и тем более широким и размытым выглядит кольцо.

\subsection{Прямая проверка гипотезы}
Приближение к объекту на расстояние, на котором космологическое расширение за время распространения сигнала становится пренебрежимо малым, должно привести к исчезновению кольцевой структуры. Наблюдатель, находящийся в непосредственной близости, зафиксирует компактное тело с чёткой поверхностью, излучающее во всех диапазонах. Никакого «горизонта событий» или «ловушки для света» обнаружено не будет.

\subsection{Временная задержка излучения от центра}
Излучение, отражённое от центральной области поверхности, достигает наблюдателя с задержкой по сравнению с излучением от периферии. При достаточно длительном мониторинге могут быть зарегистрированы последовательные кольцевые структуры, соответствующие излучению от различных зон поверхности, приходящему с разными временными задержками.

\section{Обсуждение}
Предлагаемая модель не требует введения понятий «горизонт событий», «сингулярность», «захват света гравитацией» или «искривление пространства-времени». Все наблюдаемые особенности изображений (кольцо + тёмный центр) объясняются комбинацией эффекта Хаббла и реального отражения света от физической поверхности. Гравитация выполняет в этой модели ровно одну функцию — изменение частоты излучения.

Модель даёт принципиально проверяемое предсказание: сближение с объектом должно превратить «чёрную дыру» в обычную планетоподобную структуру. Это отличает её от классической теории, предсказывающей принципиальную ненаблюдаемость горизонта.

\end{document}