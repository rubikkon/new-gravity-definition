\documentclass[12pt,a4paper]{article}
\usepackage[utf8]{inputenc}
\usepackage[T1]{fontenc}
\usepackage[english]{babel}
\usepackage{amsmath}
\usepackage{amssymb}
\usepackage{hyperref}
\usepackage{indentfirst}

\hypersetup{
    colorlinks=true,
    linkcolor=blue,
    citecolor=blue,
    urlcolor=blue
}

\title{On the Question of the Nature of Objects Observed as ``Black Holes''}

% === ЗАПОЛНИТЕ СВОИ ДАННЫЕ ЗДЕСЬ ===
\author{L.A. Serebrennikov \\
        Independent Researcher \\
        \texttt{rubikkon@gmail.com} \\[0.5em]
        \small{AI-assisted preprint for ai.vixra.org}}

\date{\today}

\begin{document}

\maketitle

\begin{abstract}
This paper proposes a model that explains the observed characteristics of compact objects in galactic centres (in particular, M87* and Sgr~A*) without invoking the concept of an event horizon or gravitational light capture. The model is based on the postulate that gravity affects exclusively the frequency of radiation (gravitational redshift) and is incapable of changing the trajectory of photons or stopping them. The observed structure ``luminous ring + dark central region'' is explained by a combination of reflection of radiation from the real surface of the object and the effects of cosmological expansion (Hubble law) acting on the light during its propagation to the observer.
\end{abstract}

\textbf{Keywords:} black holes, event horizon, gravitational redshift, Hubble expansion, light reflection, alternative gravity model

\section{Initial Postulates}

\subsection{Gravity and Light}
The gravitational field does not directly interact with light as a moving object in the sense of changing its speed or direction of motion. The only effect of gravity on radiation is a change in its frequency (gravitational redshift). A photon leaving a region with a strong gravitational field loses energy, which is registered as an increase in wavelength.

\subsection{Presence of a Surface}
The objects under study (traditionally called ``black holes'') are compact bodies possessing a material surface. The term ``event horizon'' is considered a mathematical abstraction that has no physical embodiment.

\section{Mechanism of the Observed Image Formation}

\subsection{Source of Radiation}
The radiation detected from the object is formed as a result of external light (from surrounding stars, the interstellar medium, or the object itself) incident on its surface and subsequent reflection.

\subsection{Redshift upon Reflection}
At the moment of reflection from the surface located in a strong gravitational field, the radiation experiences redshift. The magnitude of the shift depends on the gravitational potential at the point of reflection.

\subsection{Influence of Hubble Expansion during Propagation}
After reflection, the light travels toward the observer. During its propagation, cosmological expansion (the Hubble law) affects both the radiation itself and the geometric characteristics of the source object. The expansion leads to the ``stretching'' of the spatial structure from which the radiation was emitted.

\subsection{Transformation of the Beam into a Ring Structure}
Due to the stretching of the reflection region over the light travel time, the initially compact beam of radiation reflected from the central part of the surface is transformed into a ring structure by the time it reaches the observer. Radiation reflected from the peripheral zones forms the outer part of the ring. Radiation from the central region becomes ``smeared'' across the ring, creating the illusion of the absence of a signal from the center.

\subsection{Nature of the Dark Central Region}
The dark area inside the ring is not a zone devoid of radiation or a ``shadow'' cast by an invisible object. It is a region whose radiation has reached the observer in a transformed form (as part of the ring) and will be registered as part of the ring in subsequent observations with different time delays.

\section{Observational Consequences and Predictions}

\subsection{Image Dependence on Distance}
The effect of transforming a point-like reflection into a ring is a function of the distance to the object. The farther the object, the stronger the influence of Hubble expansion and the wider and more blurred the ring appears.

\subsection{Direct Test of the Hypothesis}
Approaching the object to a distance at which cosmological expansion during the signal propagation time becomes negligible should lead to the disappearance of the ring structure. An observer in immediate proximity will record a compact body with a clear surface emitting across all wavelength ranges. No ``event horizon'' or ``light trap'' will be detected.

\subsection{Time Delay of Radiation from the Center}
Radiation reflected from the central region of the surface reaches the observer with a delay compared to radiation from the periphery. With sufficiently prolonged monitoring, successive ring structures corresponding to radiation from different surface zones arriving with different time delays may be registered.

\section{Discussion}
The proposed model does not require the introduction of the concepts ``event horizon'', ``singularity'', ``gravitational capture of light'', or ``spacetime curvature''. All observed features of the images (ring + dark centre) are explained by the combination of the Hubble expansion effect and real reflection of light from a physical surface. In this model gravity performs exactly one function — changing the frequency of radiation.

The model provides a fundamentally testable prediction: approaching the object should transform the ``black hole'' into an ordinary planet-like structure. This distinguishes it from classical theory, which predicts the fundamental unobservability of the horizon.

\end{document}