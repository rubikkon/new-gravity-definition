\documentclass[12pt, a4paper]{article}
\usepackage[T2A]{fontenc}
\usepackage[utf8]{inputenc}
\usepackage[english, russian]{babel}
\usepackage{amsmath, amssymb, amsfonts}
\usepackage{graphicx}
\usepackage{hyperref}
\usepackage{geometry}
\usepackage{enumitem}
\usepackage{microtype} % Для улучшения переносов
\usepackage[normalem]{ulem} % Для подчёркивания

\geometry{left=2.5cm, right=2.5cm, top=2.5cm, bottom=2.5cm}
\hypersetup{
    unicode=true,
    pdftitle={Альтернативная гипотеза гравитации как эффекта дифференциального расширения Вселенной},
    pdfauthor={Л.~А.~Серебренников},
    pdfsubject={Теоретическая физика, космология},
    pdfkeywords={альтернативные теории гравитации, космологическое расширение, тёмная материя, параметр Хаббла, фальсифицируемость}
}

\title{Альтернативная гипотеза гравитации как эффекта дифференциального расширения Вселенной}
\author{Л.~А.~Серебренников}
\date{\today}

\begin{document}

\maketitle

% Информация об английской версии
\begingroup
\renewcommand{\thefootnote}{\fnsymbol{footnote}}
\footnotetext[1]{Английская версия статьи доступна в репозитории:\\
    \url{https://github.com/rubikkon/new-gravity-definition}}
\endgroup

\begin{abstract}
    В рамках стандартной физической парадигмы гравитация описывается либо как фундаментальное взаимодействие (закон всемирного тяготения Ньютона), либо как геометрическое свойство искривлённого пространства-времени (общая теория относительности Эйнштейна). Предлагаемая гипотеза рассматривает наблюдаемое гравитационное притяжение не как первичную силу, а как макроскопическое следствие неравномерного космологического расширения пространства на малых масштабах. Согласно модели, локальная скорость расширения модулируется плотностью энергии-импульса: области с большей плотностью (массивные тела) расширяются интенсивнее, создавая градиент расширения. Этот градиент приводит к эффективному относительному движению менее массивных тел в направлении зон повышенного расширения, что феноменологически наблюдается как притяжение. В работе излагаются основные постулаты модели, проводится их качественное сопоставление с классической динамикой, обсуждаются потенциальные следствия для проблемы тёмной материи и тёмной энергии. Центральное внимание уделяется предложению конкретных, потенциально фальсифицируемых предсказаний и путей их экспериментальной проверки с использованием современных и планируемых астрофизических инструментов.

    \textbf{Ключевые слова:} альтернативные теории гравитации, космологическое расширение, тёмная материя, параметр Хаббла, фальсифицируемость.
\end{abstract}

\section{Введение}
\label{sec:intro}

Феномен гравитации, известный со времён Ньютона и радикально переосмысленный Эйнштейном, остаётся краеугольным камнем современной физики. Несмотря на триумфальное подтверждение предсказаний общей теории относительности (ОТО) в ряде экспериментов --- от смещения перигелия Меркурия до недавних наблюдений гравитационных волн --- её интеграция с квантовой механикой, а также необходимость введения гипотетических компонентов (тёмная материя, тёмная энергия) для описания космологических данных указывают на возможную незавершённость теории.

Параллельно, космологическое расширение, открытое Хабблом и Леметром, является установленным фактом, подтверждённым наблюдениями за сверхновыми типа Ia, анизотропией реликтового излучения (миссии COBE, WMAP, Planck) и барионными акустическими осцилляциями. Современная \(\Lambda\)CDM-модель постулирует существование таинственной тёмной энергии, ответственной за ускорение этого расширения.

Данная работа предлагает синтетическую гипотезу, бросающую вызов традиционному разделению гравитации и космологии. Мы выдвигаем идею, что то, что мы интерпретируем как гравитационное притяжение на масштабах Солнечной системы и галактик, может быть реинтерпретировано как проявление локальных неоднородностей в скорости расширения пространства. В этой картине массивные объекты не ``притягивают'', а, образно говоря, активнее ``отталкивают'' пространство вокруг себя, создавая градиент, который и управляет движением тел.

\section{Основной постулат и качественная модель}
\label{sec:postulate}

Рассмотрим однородную и изотропную Вселенную, описываемую метрикой Фридмана--Робертсона--Уокера (FLRW) с масштабным фактором \(a(t)\). Скорость расширения характеризуется параметром Хаббла \(H(t) = \dot{a}/a\). В стандартной космологии \(H\) зависит от глобальной средней плотности энергии Вселенной \(\bar{\rho}(t)\).

Основной постулат предлагаемой гипотезы заключается в следующем: \emph{локальная мгновенная скорость расширения пространства в данной точке является функцией локальной плотности энергии-импульса \(\rho(\mathbf{x}, t)\) в этой точке}. Простейшая линейная форма этой зависимости может быть записана как:
\begin{equation}
    \label{eq:main}
    H_{\text{лок}}(\mathbf{x}, t) = H_0(t) + \kappa \cdot \rho(\mathbf{x}, t),
\end{equation}
где:
\begin{itemize}[noitemsep,topsep=0pt]
    \item \(H_{\text{лок}}(\mathbf{x}, t)\) --- эффективный локальный параметр Хаббла,
    \item \(H_0(t)\) --- фоновое значение, соответствующее расширению Вселенной в отсутствие локальных неоднородностей,
    \item \(\rho(\mathbf{x}, t)\) --- плотность энергии (включая массу покоя, давление, иные формы энергии),
    \item \(\kappa\) --- положительная фундаментальная постоянная (размерность \([ \kappa ] = T M^{-1} L^{-3}\)), связывающая плотность с вкладом в расширение.
\end{itemize}

Таким образом, область пространства, содержащая массивное тело (звезду, планету), характеризуется повышенным значением \(H_{\text{лок}}\) по сравнению с пустым пространством вокруг. Рассмотрим два пробных тела \(A\) и \(B\), находящихся в таком неоднородном поле расширения (рис.~\ref{fig:gradient}).

\begin{figure}[htbp]
    \centering
    \begin{minipage}{0.6\textwidth}
        \centering
        \includegraphics[width=\linewidth]{example-image} % Замените на свой рисунок
        \caption{Схематическое изображение градиента локального параметра Хаббла \(H_{\text{лок}}\) вокруг массивного тела \(M\). Пробные тела \(A\) и \(B\) испытывают дифференциальное расширение, эффективно сближаясь с \(M\).}
        \label{fig:gradient}
    \end{minipage}
\end{figure}

Расстояние между пробным телом и массивным объектом изменяется под действием двух эффектов:
\begin{enumerate}[noitemsep]
    \item Глобального космологического расширения, стремящегося их разнести.
    \item Локального дифференциального расширения: пространство между телом \(A\) и массой \(M\) расширяется со средней скоростью, меньшей, чем пространство на противоположной от \(M\) стороне тела \(A\). Это создаёт \emph{эффект относительного смещения} тела \(A\) в сторону области с бо\'{о}льшим \(H_{\text{лок}}\), то есть в сторону массы \(M\).
\end{enumerate}

Количественно, для малых расстояний \(r\) от точечной массы \(M\) и в пределе слабых полей можно ожидать, что результирующая относительная скорость сближения будет пропорциональна градиенту \(H_{\text{лок}}\) и, следовательно, градиенту плотности \(\nabla \rho\). Для статического распределения массы \(\rho \sim M \delta(\mathbf{r})\) это приводит к силе, обратно пропорциональной квадрату расстояния, аналогичной ньютоновской.

\section{Качественное согласование с известными явлениями}
\label{sec:phenomenology}

\subsection{Ньютоновский предел}

Вдали от массивного тела \(\rho \to 0\) и \(H_{\text{лок}} \approx H_0\). Вблизи тела \(H_{\text{лок}}\) возрастает. Пробное тело, изначально покоящееся относительно фонового расширения, будет иметь скорость относительно массивного тела, определяемую разностью \(H_{\text{лок}}(r) \cdot r - H_0 \cdot r = \kappa \rho(r) \cdot r\). Для точечной массы \(\rho(r)\) эффективно падает как \(1/r^3\) (в смысле усреднения по малому объёму), что даёт зависимость скорости от \(1/r^2\). Это соответствует ньютоновскому ускорению, где роль гравитационной постоянной \(G\) играет комбинация \(\kappa\), \(H_0\) и, возможно, других параметров.

\subsection{Гравитационное линзирование}

Если гравитация есть проявление искривления пространства-времени в ОТО, то линзирование --- его прямое следствие. В рамках предлагаемой модели искривление пространства не постулируется. Однако изменение скорости расширения влияет на эффективный показатель преломления для света. Области с повышенным \(H_{\text{лок}}\) (а значит, и повышенной \(\rho\)) будут иметь отличные от вакуума свойства, что может приводить к отклонению световых лучей, качественно аналогичному линзированию. Формализация этого эффекта требует построения полной метрической теории.

\subsection{Чёрные дыры и горизонты событий}

В ОТО чёрная дыра возникает как сингулярность в решении уравнений. В нашей модели экстремально высокая плотность должна приводить к экстремально высокому локальному расширению. Можно предположить, что существует критическая плотность, при которой локальная скорость расширения пространства вблизи объекта превышает скорость света относительно удалённого наблюдателя. Это создало бы область, из которой сигналы не могут выйти наружу --- аналог горизонта событий, но с иной физической природой.

\section{Потенциальные следствия: переосмысление тёмной материи и тёмной энергии}
\label{sec:consequences}

\subsection{Проблема плоскостности ротационных кривых галактик}

В стандартной модели аномально высокая скорость вращения звёзд на периферии спиральных галактик объясняется наличием гало тёмной материи. В нашей гипотезе центральная область галактики (балдж, бар), обладающая высокой плотностью видимой материи, создаёт градиент \(H_{\text{лок}}\), простирающийся далеко за пределы видимого диска. Звезда на периферии находится в области, где фоновое \(H_0\) мало, но градиент, создаваемый всей видимой массой галактики, ещё существенен. Это может приводить к дополнительной тангенциальной (или эффективно центростремительной) компоненте скорости, не требующей введения невидимой массы.

\subsection{Кластеризация и скопления галактик}

Наблюдаемые скорости галактик в скоплениях, такие как в скоплении Кома, также считаются аномально высокими без тёмной материи. Если рассматривать скопление как единый объект с высокой суммарной плотностью \(\rho_{\text{скоп}}\), то создаваемый им градиент \(H_{\text{лок}}\) может простираться на весь объём скопления, влияя на динамику его членов.

\subsection{Тёмная энергия и ускоренное расширение}

Уравнение (\ref{eq:main}) предлагает иной взгляд на ускорение расширения Вселенной. Если \(\kappa > 0\), то сама плотность энергии (включая обычную и, возможно, новые поля) вносит положительный вклад в \(H_{\text{лок}}\). В поздней Вселенной, когда материя становится разреженной, доминирующий вклад может давать некая остаточная плотность энергии вакуума \(\rho_{\Lambda}\), постоянная во времени и пространстве. Тогда \(H_{\text{лок}} \approx H_0 + \kappa \rho_{\Lambda} = \text{const}\), что соответствует де-ситтеровскому расширению с ускорением. Таким образом, ``тёмная энергия'' может быть не отдельной сущностью, а проявлением того же механизма, через константу \(\rho_{\Lambda}\).

\section{Возможные пути фальсификации гипотезы}
\label{sec:falsification}

Гипотеза имеет смысл только если предлагает проверяемые отличия от ОТО/\(\Lambda\)CDM. Ниже приведены потенциальные пути её экспериментальной фальсификации.

\subsection{Локальные тесты в Солнечной системе}

ОТО с высочайшей точностью подтверждена в пределах Солнечной системы. Любая жизнеспособная альтернатива должна воспроизводить эти результаты.
\begin{enumerate}
    \item \textbf{Дополнительные перигельные прецессии:} Модель должна количественно предсказать прецессию перигелия Меркурия, точно совпадающую с данными (43'' за столетие). Это накладывает жёсткие ограничения на связь между \(\kappa\), \(H_0\) и эффективной \(G\).
    \item \textbf{Лазерная локация Луны (LLR):} Чрезвычайно точные измерения расстояния до Луны (точность $\sim$1 см) чувствительны к любым отклонениям от законов Ньютона/ОТО. Гипотеза должна предсказывать отсутствие аномалий в данных LLR или объяснять их в рамках своих параметров.
    \item \textbf{Полёт космических аппаратов:} Траектории зондов <<Пионер>>, <<Вояджер>>, <<Новые горизонты>> тщательно отслеживаются. Легендарная <<аномалия Пионеров>> (ныне объяснённая тепловым изрывлением) --- пример уровня точности, на котором может проявиться расхождение.
\end{enumerate}

\subsection{Астрофизические наблюдения}

\begin{enumerate}[resume]
    \item \textbf{Структура и эволюция галактик:} Модель должна быть способна воспроизвести разнообразие наблюдаемых ротационных кривых без подгонки профилей тёмной материи. Ключевым тестом будет предсказание связи между видимым распределением массы (по светимости и газу) и детальной формой ротационной кривой для широкого класса галактик.
    \item \textbf{Гравитационное линзирование в скоплениях:} Измерения линзирования в скоплениях (например, скопление Пуля) позволяют независимо картографировать распределение массы. Наша модель предсказывает, что карта массы, восстановленная по линзированию, должна в деталях соответствовать распределению видимой (барионной) массы, а не гладкому гало тёмной материи. Сильные расхождения станут фальсификацией.
    \item \textbf{Космологические тесты:} Модель должна давать конкретные предсказания для:
          \begin{itemize}
              \item Спектра мощности флуктуаций реликтового излучения (CMB).
              \item Скорости роста космических структур (подавление роста на больших масштабах может отличаться от \(\Lambda\)CDM).
              \item Соотношения между расстоянием и красным смещением от сверхновых Ia, которое может стать нелинейным из-за локальных вариаций \(H_{\text{лок}}\).
          \end{itemize}
\end{enumerate}

\subsection{Лабораторные эксперименты}

\begin{enumerate}[resume]
    \item \textbf{Измерения гравитационной постоянной \(G\):} Если гравитация --- производный эффект, то эффективная <<константа>> \(G_{\text{эфф}}\) может зависеть от локальной плотности окружающей среды или истории расширения. Высокоточные эксперименты по измерению \(G\) (например, с помощью торсионных балансов или атомных интерферометров) в различных условиях (под землёй, на разной высоте, в вакуумированных камерах разного размера) могут выявить такие вариации.
    \item \textbf{Поиск зависимости от времени:} В модели, где \(H_0(t)\) меняется со временем, может меняться и эффективная сила <<притяжения>>. Сверхточные долговременные измерения орбит планет или параметров двойных пульсаров могут наложить пределы на такую зависимость.
\end{enumerate}

\section{Критика и открытые вопросы}
\label{sec:critique}

Гипотеза сталкивается с рядом серьёзных теоретических и практических вызовов:
\begin{itemize}
    \item \textbf{Принцип эквивалентности:} Классические эксперименты (Этвёша и др.) с высочайшей точностью подтверждают равенство инертной и гравитационной масс. Любая альтернативная теория должна объяснять это равенство. В нашей модели оно может следовать из того, что и инертная масса (через \(\rho\)), и <<гравитационный заряд>> определяются одной и той же величиной --- плотностью энергии.
    \item \textbf{Векторная/скалярная природа:} Гравитация в ОТО --- тензорное взаимодействие (2-спиновый гравитон). Предлагаемый механизм, основанный на скалярном поле \(H_{\text{лок}}(\rho)\), по своей природе ближе к скалярным теориям, которые имеют известные проблемы (например, с поляризациями гравитационных волн). Необходимо показать, как модель воспроизводит тензорный характер гравитационного излучения, наблюдаемый LIGO.
    \item \textbf{Каузальность:} В ОТО распределение массы влияет на геометрию, которая определяет движение. В нашей модели локальная плотность мгновенно (?) определяет локальное расширение. Это может вступать в конфликт с принципами специальной теории относительности. Требуется релятивистски инвариантная формулировка, возможно, с введением динамического скалярного поля \(\phi(x^\mu)\), модулирующего расширение.
    \item \textbf{Формализация:} Гипотеза нуждается в строгой математической формулировке в рамках модифицированной гравитации (например, \(f(H, \rho)\)-теории) или эффективной теории поля. Только тогда можно будет делать точные количественные предсказания.
\end{itemize}

\section{Заключение}
\label{sec:conclusion}

Представленная гипотеза предлагает радикальный сдвиг парадигмы: гравитационное <<притяжение>> реинтерпретируется как макроскопическое следствие дифференциального космологического расширения, индуцируемого локальными неоднородностями плотности энергии. Эта идея, хотя и является в высшей степени спекулятивной, обладает рядом потенциальных преимуществ:
\begin{itemize}
    \item Объединяет в рамках одного концептуального механизма феномены, приписываемые гравитации и тёмной энергии.
    \item Предлагает альтернативу гипотетической тёмной материи для объяснения аномалий в динамике галактик.
    \item Смещает фокус с геометрии пространства-времени на динамику его расширения.
\end{itemize}

Однако ценность гипотезы определяется не её интуитивной привлекательностью, а \emph{фальсифицируемостью}. В работе предложен ряд конкретных тестов --- от прецизионной астрометрии в Солнечной системе до анализа данных линзирования в скоплениях галактик и лабораторных измерений постоянной \(G\). Эти тесты способны либо опровергнуть модель в её простейшей форме, либо указать направления для её уточнения.

Путь от качественной идеи до полноценной научной теории долог и требует построения последовательной математической модели, способной давать однозначные количественные предсказания для всего множества существующих экспериментальных данных. Тем не менее, сама постановка вопроса о возможной связи локальной гравитации и глобального расширения может стимулировать новые исследовательские программы и поиск аномалий в данных следующего поколения, которые не укладываются в рамки стандартной космологической модели.

\section*{Благодарности}
Автор выражает признательность коллегам за полезные обсуждения и критические замечания. Работа выполнена в рамках личной исследовательской инициативы.

\vspace{1cm}

\noindent \textbf{Информация о версиях:} Английская версия данной работы доступна в репозитории: \url{https://github.com/rubikkon/new-gravity-definition}

\end{document}