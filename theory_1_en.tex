\documentclass[12pt,a4paper]{article}
\usepackage[utf8]{inputenc}
\usepackage[T1]{fontenc}
\usepackage[english]{babel}
\usepackage{amsmath, amssymb}
\usepackage{geometry}
\usepackage{array}
\geometry{top=2cm, bottom=2cm, left=2.5cm, right=1.5cm}

% Added fields for professional publication:
% - Abstract: Brief description of the work for viXra (required for submission).
% - Keywords: 5-10 words for indexing (example filling).
% - Author: Specify real name or pseudonym (example: L. A. Serebrennikov).
% - Date: Current date or submission date (example: February 2026).
% - Category for viXra: For AI section, e.g., "Artificial Intelligence" or "Physics" (specify in title or as metadata).
% - MSC/AMSC codes: Optional for mathematical works (example: 83Cxx for gravity/expansion).
% Do not change format: Keep tabular, itemize, etc. as is.
% For viXra: Compile to PDF, submit with metadata (title, abstract, authors, keywords, category: AI).

\title{Model of Local Expansion \\ and Gravity as Interaction of Potentials}
\author{L. A. Serebrennikov \\ % Example filling: Specify your name or affiliation, e.g., Independent Researcher, Kawaguchi, Japan
\small Independent Researcher \\ % Example affiliation
\small Email: rubikkon@gmail.com} % Example email for contact
\date{February 12, 2026} % Example filling: Month and year of publication

\begin{document}

\maketitle

\begin{abstract}
This work presents a model of the Universe's expansion and gravity through an analogy with rubber balloons. Key terms are defined, such as mass as expansion potential, volume as the degree of matter stretching, and gravity as the interaction of potentials against the background of expansion. The model describes expansion as a search for equilibrium, where gravity arises from competition for space. A mathematical formulation is proposed with the Hubble parameter as a function of density and potential. % Example filling: Brief overview (100-200 words) for viXra.
\end{abstract}

\textbf{Keywords:} Universe expansion, gravity, Hubble parameter, energy potential, matter density, balloon analogy, search for equilibrium. % Example filling: 5-10 keywords for search on viXra.

\textbf{viXra Category:} Artificial Intelligence / Physics (Cosmology). % Example: Specify when submitting to ai.vixra.org.

\textbf{MSC codes:} 83C05 (General relativity), 85A40 (Cosmology). % Example: Optional for mathematical classification.

\section*{Preface}
This text is a direct record of the model, presented through an analogy with rubber balloons.
All definitions are given as they were formulated.

\section{WHAT IS WHAT (Dictionary of Terms)}

\begin{center}
\begin{tabular}{|p{5cm}|p{10cm}|}
\hline
\textbf{Term} & \textbf{Definition} \\
\hline
Deflated rubber balloon & Area of space with maximum energy density. Not expanded. Has no visible internal volume — compressed to the limit. Its rubber is the energy of bonds, ready for stretching. \\
\hline
Balloon mass & Not the amount of substance. This is the potential energy for expansion in volume. Stored ability to create maximum volume. \\
\hline
Balloon volume & Characteristic not of space, but of matter itself. Degree of its stretching, density. Fully deflated balloon = minimum volume, maximum density. Fully inflated balloon = maximum volume, minimum density. \\
\hline
Inflating the balloon & Hubble parameter. Expansion of the Universe. \\
\hline
Two balloons nearby & Two areas of space with different masses (potentials) and different volumes, expanding near each other. \\
\hline
Interaction during inflation & Gravity. Interaction of rest masses against the background of the Universe's expansion. \\
\hline
Expansion of the Universe & Search for equilibrium state. \\
\hline
\end{tabular}
\end{center}

\section{LOGIC OF THE PROCESS (How it works)}

\subsection{Initial state}
Each deflated balloon has:
\begin{itemize}
    \item mass $M$ — expansion potential;
    \item minimum volume $V_{\min}$;
    \item maximum density $\rho_{\max} = M / V_{\min}$.
\end{itemize}

The greater the mass, the larger the maximum volume that can be achieved:
\[
V_{\max} \sim M
\]

\subsection{Expansion (inflation)}
All balloons start inflating simultaneously.
The inflation rate of each balloon depends on its current density, potential, and current volume.

Hubble parameter $H = \dot{V} / V$ is a function:
\[
H = f(\rho, M, V)
\]

\subsection{Interaction (gravity)}
When two balloons inflate nearby, they cannot expand independently.
They are in a common rubber medium. Their expansion competes for space.

Rule:
\begin{itemize}
    \item Balloon with greater mass has more "right" to volume;
    \item Balloon with lesser mass is forced to expand slower in the direction of the larger one;
    \item This is perceived as attraction of the smaller to the larger.
\end{itemize}

Gravity is the redistribution of expansion in favor of the greater potential.

\subsection{Expansion as search for equilibrium}
The Universe strives for a state where:
\begin{itemize}
    \item All balloons are inflated to their maximum volume;
    \item Densities are equalized to minimum;
    \item Density gradients are absent;
    \item Expansion is stopped ($H = 0$);
    \item Gravity is absent;
    \item Mass equals zero.
\end{itemize}

This state is the absolute zero of expansion.
It is never achieved, because as long as there are gradients — there is expansion, and as long as there is expansion — there are inhomogeneities.

\end{document}