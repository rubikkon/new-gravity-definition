\documentclass[12pt,a4paper]{article}
\usepackage[utf8]{inputenc}
\usepackage[T2A]{fontenc}
\usepackage[russian]{babel}
\usepackage{amsmath, amssymb}
\usepackage{geometry}
\usepackage{array}
\geometry{top=2cm, bottom=2cm, left=2.5cm, right=1.5cm}

% Добавленные поля для профессиональной публикации:
% - Абстракт (abstract): Краткое описание работы для viXra (требуется при подаче).
% - Ключевые слова (keywords): 5-10 слов для индексации (пример заполнения).
% - Автор: Укажите реальное имя или псевдоним (пример: L. A. Serebrennikov).
% - Дата: Текущая дата или дата подачи (пример: February 2026).
% - Категория для viXra: Для AI раздела, например, "Artificial Intelligence" или "Physics" (укажите в заголовке или как метаданные).
% - MSC/AMSC codes: Опционально для математических работ (пример: 83Cxx для гравитации/расширения).
% Не меняем формат: Оставляем tabular, itemize и т.д. как есть.
% Для viXra: Скомпилируйте в PDF, подайте с метаданными (title, abstract, authors, keywords, category: AI).

\title{Модель локального расширения \\ и гравитации как взаимодействия потенциалов}
\author{L. A. Serebrennikov \\ % Пример заполнения: Укажите ваше имя или аффилиацию, e.g., Independent Researcher, Kawaguchi, Japan
\small Independent Researcher \\ % Пример аффилиации
\small Email: rubikkon@gmail.com} % Пример email для контакта
\date{12 Февраля 2026} % Пример заполнения: Месяц и год публикации

\begin{document}

\maketitle

\begin{abstract}
Эта работа представляет модель расширения Вселенной и гравитации через аналогию с резиновыми шариками. Определяются ключевые термины, такие как масса как потенциал расширения, объём как степень растяжения материи, и гравитация как взаимодействие потенциалов на фоне расширения. Модель описывает расширение как поиск равновесия, где гравитация возникает из конкуренции за пространство. Предлагается математическая формулировка с параметром Хаббла как функцией плотности и потенциала. % Пример заполнения: Краткий обзор (100-200 слов) для viXra.
\end{abstract}

\textbf{Ключевые слова:} расширение Вселенной, гравитация, параметр Хаббла, потенциал энергии, плотность материи, аналогия шариков, поиск равновесия. % Пример заполнения: 5-10 ключевых слов для поиска на viXra.

\textbf{Категория viXra:} Artificial Intelligence / Physics (Cosmology). % Пример: Укажите при подаче на ai.vixra.org.

\textbf{MSC codes:} 83C05 (General relativity), 85A40 (Cosmology). % Пример: Опционально для математической классификации.

\section*{Предисловие}
Этот текст — прямая запись модели, изложенной через аналогию с резиновыми шариками.
Все определения даны так, как они были сформулированы.

\section{ЧТО ЕСТЬ ЧТО (Словарь терминов)}

\begin{center}
\begin{tabular}{|p{5cm}|p{10cm}|}
\hline
\textbf{Термин} & \textbf{Определение} \\
\hline
Сдутый резиновый шарик & Область пространства с максимальной плотностью энергии. Не расширен. Не имеет видимого внутреннего объёма — сжат до предела. Его резина — энергия связей, готовая к растяжению. \\
\hline
Масса шарика & Не количество вещества. Это потенциал энергии к расширению в объёме. Запасённая способность создать максимальный объём. \\
\hline
Объём шарика & Характеристика не пространства, а самой материи. Степень её растяжения, плотность. Полностью сдутый шарик = минимальный объём, максимальная плотность. Полностью надутый шарик = максимальный объём, минимальная плотность. \\
\hline
Надувание шарика & Параметр Хаббла. Расширение Вселенной. \\
\hline
Два шарика рядом & Две области пространства с разными массами (потенциалами) и разными объёмами, расширяющиеся вблизи друг друга. \\
\hline
Взаимодействие при надувании & Гравитация. Взаимодействие масс покоя на фоне расширения Вселенной. \\
\hline
Расширение Вселенной & Поиск состояния равновесия. \\
\hline
\end{tabular}
\end{center}

\section{ЛОГИКА ПРОЦЕССА (Как это работает)}

\subsection{Исходное состояние}
Каждый сдутый шарик обладает:
\begin{itemize}
    \item массой $M$ — потенциалом расширения;
    \item минимальным объёмом $V_{\min}$;
    \item максимальной плотностью $\rho_{\max} = M / V_{\min}$.
\end{itemize}

Чем больше масса, тем больший максимальный объём может быть достигнут:
\[
V_{\max} \sim M
\]

\subsection{Расширение (надувание)}
Все шарики начинают надуваться одновременно.
Скорость надувания каждого шарика зависит от его текущей плотности, потенциала и текущего объёма.

Параметр Хаббла $H = \dot{V} / V$ есть функция:
\[
H = f(\rho, M, V)
\]

\subsection{Взаимодействие (гравитация)}
Когда два шарика надуваются рядом, они не могут расширяться независимо.
Они находятся в общей резиновой среде. Их расширение конкурирует за пространство.

Правило:
\begin{itemize}
    \item шарик с большей массой имеет больше «права» на объём;
    \item шарик с меньшей массой вынужден расширяться медленнее в направлении большого;
    \item это воспринимается как притяжение малого к большому.
\end{itemize}

Гравитация — это перераспределение расширения в пользу большего потенциала.

\subsection{Расширение как поиск равновесия}
Вселенная стремится к состоянию, где:
\begin{itemize}
    \item все шарики надуты до своего максимального объёма;
    \item плотности выровнены до минимальной;
    \item градиенты плотности отсутствуют;
    \item расширение остановлено ($H = 0$);
    \item гравитация отсутствует;
    \item масса равна нулю.
\end{itemize}

Это состояние — абсолютный нуль расширения.
Оно никогда не достигается, потому что пока есть градиенты — есть расширение, а пока есть расширение — есть неоднородности.

\end{document}