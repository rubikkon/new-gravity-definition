% !TEX spellcheck = off
\documentclass[12pt, a4paper]{article}
\usepackage[T2A]{fontenc}
\usepackage[utf8]{inputenc}
\usepackage[english,russian]{babel}
\usepackage{amsmath, amssymb, amsfonts}
\usepackage{graphicx}
\usepackage{hyperref}
\usepackage{xurl}
\usepackage{geometry}
\usepackage{enumitem}
\usepackage[normalem]{ulem}
\usepackage{setspace}
\onehalfspacing
\geometry{left=2.5cm, right=2.5cm, top=2.5cm, bottom=2.5cm}
\hypersetup{
    unicode=true,
    pdftitle={Альтернативная гипотеза гравитации как эффекта дифференциального расширения Вселенной},
    pdfauthor={Леонид Серебренников},
    pdfsubject={Теоретическая физика, космология},
    pdfkeywords={альтернативная гравитация, космологическое расширение, тёмная материя,
                параметр Хаббла, фальсифицируемость}
}

\title{Альтернативная гипотеза гравитации как эффекта дифференциального расширения Вселенной}
\author{Леонид Серебренников\thanks{Автор для переписки: rubikkon@gmail.com}}
\date{}

\begin{document}

\maketitle

\begin{abstract}
    В стандартной модели гравитация описывается либо как фундаментальное взаимодействие
    (закон всемирного тяготения Ньютона), либо как геометрическое свойство пространства-времени
    (общая теория относительности Эйнштейна). Предлагаемая здесь гипотеза рассматривает
    наблюдаемое гравитационное притяжение как следствие неоднородного космологического расширения.
    Массивные объекты, обладая большей плотностью энергии, вызывают более интенсивное локальное
    расширение пространства, что приводит к эффективному отталкиванию менее массивных тел в направлении
    зон повышенного расширения. В статье изложены основные постулаты модели, её возможные следствия
    для проблемы тёмной материи и предложены конкретные пути экспериментальной фальсификации.
\end{abstract}

\section*{Примечание о языке}
Английская и русская версии статьи доступны в репозитории:
\begin{center}
\url{https://github.com/rubikkon/new-gravity-definition}
\end{center}
Русская версия содержит оригинальную формулировку данной гипотезы.

\vspace{0.5cm}
\noindent \textbf{Ключевые слова:} альтернативная гравитация, космологическое расширение, 
тёмная материя, дифференциальное расширение, фальсифицируемость.

\section{Введение}
\label{sec:intro}

Космологическое расширение подтверждено эмпирически: закон Хаббла--Леметра \cite{hubble},
данные о сверхновых типа Ia и анизотропия реликтового микроволнового излучения 
(миссии COBE, WMAP, Planck \cite{planck}). Ускорение расширения в современной 
космологии приписывается тёмной энергии (член $\Lambda$ в уравнениях Фридмана). 
В настоящей работе предлагается гипотеза, согласно которой наблюдаемое 
гравитационное притяжение на масштабах Солнечной системы и галактик может быть 
переосмыслено как макроскопическое проявление локальных вариаций скорости этого 
расширения.

\section{Основной постулат}
\label{sec:postulate}

Гипотеза постулирует зависимость локальной скорости расширения
(производной масштабного фактора) от локальной плотности
энергии $\rho$ в данной области пространства-времени:
\begin{equation}
    \label{eq:main}
    \dot{H}_{\text{loc}} = H_0 + \kappa \rho,
\end{equation}
где $\dot{H}_{\text{loc}}$ — локальная производная параметра 
Хаббла (или масштабного фактора), $H_0$ — глобальная 
постоянная Хаббла, характеризующая фоновое расширение 
Вселенной, $\kappa$ — феноменологический коэффициент, 
связывающий плотность энергии с вкладом в расширение.

Возникающий градиент $\dot{H}_{\text{loc}}$ создаёт эффективную силу, 
направленную из областей меньшей плотности энергии в области большей 
плотности. На макроскопическом уровне это явление феноменологически 
эквивалентно ньютоновскому притяжению, хотя его микроскопический механизм
имеет противоположный характер (эффективное <<отталкивание>> из зон 
меньшего расширения в зоны большего расширения).

\section{Возможные следствия: проблема тёмной материи}
\label{sec:consequences}

Данная модель предлагает альтернативное объяснение аномалий в кривых 
вращения галактик, обычно приписываемых наличию тёмной материи. 
Согласно гипотезе, повышенное расширение в центральных областях галактик 
(где выше плотность звёзд и газа) может влиять на динамику периферийных 
звёзд, увеличивая их тангенциальную скорость. В стандартной интерпретации 
это наблюдается как наличие невидимой массы, создающей дополнительное 
гравитационное поле.

\section{Возможные пути фальсификации}
\label{sec:falsification}

Для проверки предложенной гипотезы могут быть использованы следующие 
экспериментальные и наблюдательные методы:

\begin{enumerate}
    \item \textbf{Прецизионная астрометрия.} Проекты вроде миссии 
          \textit{Gaia} \cite{gaia} или будущие космические интерферометры 
          могут выявить малые отклонения в движении тел Солнечной системы 
          или двойных звёзд от предсказаний общей теории относительности.
          Согласно гипотезе, эти отклонения должны коррелировать с 
          локальными значениями параметра $H_0$.

    \item \textbf{Анализ гравитационных волн.} Детекторы LIGO \cite{ligo}, 
          Virgo, KAGRA и будущая миссия LISA могут обнаружить слабые 
          космологические поправки к фазе и амплитуде сигналов от слияний 
          компактных объектов на больших красных смещениях, вызванные 
          зависимостью гравитационного взаимодействия от локальной 
          плотности энергии.

    \item \textbf{Космологические обзоры.} Данные миссий \textit{Euclid},
          \textit{Roman Space Telescope} или LSST позволят исследовать 
          возможные корреляции между распределением видимой материи и 
          локальными вариациями параметра расширения на масштабах 
          10--100 Мпк.

    \item \textbf{Лабораторные эксперименты.} Высокоточные измерения с 
          использованием атомных часов или модифицированных установок 
          типа Кавендиша могут быть направлены на обнаружение возможной 
          зависимости эффективной <<постоянной>> $G$ от плотности 
          окружающей среды или её потенциальных временных вариаций.
\end{enumerate}

\section{Заключение и обсуждение}
\label{sec:conclusion}

Представленная гипотеза рассматривает гравитацию не как фундаментальное 
взаимодействие, а как следствие дифференциального космологического 
расширения, модулируемого распределением плотности энергии. Это открывает 
путь к концептуальному объединению гравитации и явлений космологического 
ускорения (в отличие от подхода общей теории относительности \cite{einstein})
и потенциально устраняет необходимость в гипотетической тёмной материи 
для объяснения кинематики галактик.

Однако гипотеза требует дальнейшей серьёзной разработки, включая:
\begin{itemize}
    \item Математическую формализацию в рамках модифицированной метрики 
          или эффективной теории поля.
    \item Количественный вывод ньютоновского и постньютоновского пределов.
    \item Строгую согласованность с полным набором высокоточных тестов 
          общей теории относительности в Солнечной системе.
\end{itemize}

На данный момент гипотеза остаётся спекулятивной и не имеет прямого 
эмпирического подтверждения. Её ценность заключается в предложении нового 
направления исследований и конкретных, потенциально фальсифицируемых 
предсказаний для следующего поколения астрофизических и лабораторных 
экспериментов.

\section*{Благодарности}
Автор благодарит коллег из отдела теоретической физики за ценные обсуждения.
Работа выполнена в рамках личной исследовательской инициативы.

\vspace{0.5cm}

\noindent \textbf{Информация о версии:} Русская версия работы доступна в
репозитории: \url{https://github.com/rubikkon/new-gravity-definition}

\begin{thebibliography}{99}
    \bibitem{hubble} Hubble, E.~P. A relation between distance and radial 
          velocity among extra-galactic nebulae. \textit{Proceedings of the 
          National Academy of Sciences}, 15(3):168--173, 1929.

    \bibitem{planck} Planck Collaboration. Planck 2018 results. VI. 
          Cosmological parameters. \textit{Astronomy \& Astrophysics}, 
          641:A6, 2020.

    \bibitem{gaia} Gaia Collaboration. The Gaia mission. \textit{Astronomy 
          \& Astrophysics}, 595:A1, 2016.

    \bibitem{ligo} Abbott, B.~P. et al. Observation of gravitational waves 
          from a binary black hole merger. \textit{Physical Review Letters}, 
          116(6):061102, 2016.

    \bibitem{einstein} Einstein, A. The foundation of the general theory of 
          relativity. \textit{Annalen der Physik}, 354(7):769--822, 1916.
\end{thebibliography}

\end{document}